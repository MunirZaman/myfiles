\documentclass[11pt,numbers=noenddot,svgnames,dvipsnames]{scrartcl}
\usepackage[top=1in, left=1in, right=1in, bottom=1in]{geometry}
\usepackage[page, head, date, asy]{munir}

\title{Cevians}
\author{Munir Uz Zaman}
\date{Date: \today}

\begin{document}
\maketitle

\section{Cevians}
\begin{definition*}
    Let $\triangle ABC$ be a triangle and let $X$ be a point on $BC$. 
    The segment $AX$ is called a \vocab{cevian} of triangle $ABC$.
\end{definition*}
\begin{center}
\begin{asy}
size(5cm);
pair A, B, C, X;
A = (2, 3);
B = (0, 0);
C = (4, 0);
X = (B + C)*(0.65);

D(A -- B -- C -- cycle);
D(A -- X);
dot("$A$", A, align=N);
dot("$B$", B, align=SW);
dot("$C$", C, align=SE);
dot("$X$", X, align=S);
\end{asy}
\end{center}

\begin{theorem}
    If $ABC$ is a triangle and $X$ is a point on $BC$ then 
    \[
        \frac{BX}{XC} = \frac{[ABX]}{[ACX]}
    \]
\end{theorem}

\begin{corollary}
    \[
        \frac{BX}{XC} = \frac{AB \times \sin \angle BAX}{AC \times \sin \angle CAX}
    \]
\end{corollary}
\begin{proof}
    Since 
    \begin{align*}
        & [ABX] = \frac{1}{2}AB \times AX \times \sin \angle BAX \\
        & [ACX] = \frac{1}{2}AC \times AX \times \sin \angle CAX
    \end{align*}
    we have
    \begin{align*}
        \frac{BX}{XC} = \frac{[ABX]}{[ACX]} = \frac{AB \times \sin \angle BAX}{AC \times \sin \angle CAX}
    \end{align*}
\end{proof}

\begin{corollary}
    \[
        \frac{AB}{AC} =  \frac{\sin \angle ACB}{\sin \angle ABC}
    \]
\end{corollary}
\begin{proof}
    \begin{align*}
        \frac{BX}{XC} = \frac{AB \times BX \times \sin \angle ABC}{AC \times XC \times \sin \angle ACB} \implies 
        \frac{AB}{AC} =  \frac{\sin \angle ACB}{\sin \angle ABC}
    \end{align*}
\end{proof}

\begin{corollary}[Angle Bisector Theorem]
    If $AX$ is the angle bisector of $\angle BAC$ then 
    \[
        \frac{BX}{CX} = \frac{BA}{CA}
    \]
\end{corollary}

\begin{problem}
    $ABCD$ is a square with $AB = 1$ and $M, N$ are the midpoints of $BC$ and $CD$. 
    $P, Q$ are the intersection of $BD, AM$ and $BD, AN$. Find the length of $PQ$. 
    \begin{center}
    \begin{asy}
    size(6cm);
    pair A, B, C, D, M, N, P, Q;
    A = (0, 2);
    B = (0, 0);
    C = (2, 0);
    D = (2, 2);
    M = midpoint(B -- C);
    N = midpoint(C -- D);
    P = extension(A, M, B, D);
    Q = extension(A, N, B, D);

    D(A -- B -- C -- D -- cycle);
    D(B -- D);
    D(A -- M);
    D(A -- N);

    dot("$A$", A, align=NW);
    dot("$B$", B, align=SW);
    dot("$C$", C, align=SE);
    dot("$D$", D, align=NE);
    dot("$M$", M, align=S);
    dot("$N$", N, align=E);
    dot("$P$", P, align=E);
    dot("$Q$", Q, align=S);
    \end{asy}
    \end{center}
\end{problem}
\begin{sol}
    Since $AP$ is a cevian of triangle $ABD$, we have 
    \[
        \frac{PD}{BP} = \frac{\sin \angle PAD}{\sin \angle PAB}
                      = \frac{\sin 90\dg - \angle PAB}{\sin \angle PAB} 
                      = \frac{\cos \angle PAB}{\sin \angle PAB} 
                      = \cot \angle PAB
    \]
    Since $\triangle ABM$ is a right triangle, we have
    \[
        \cot \angle PAB = \frac{AB}{BM} = 2
    \]
    Therefore 
    \[
        \frac{PD}{BP} = 2 \implies \frac{QD + PQ}{PB} = 2 \implies \frac{PQ}{PB} = 1 \implies BP = PQ = QD
    \]
    Hence 
    \[
        PQ = \frac{\sqrt{2}}{3}
    \]
\end{sol}

\begin{problem}[All-Russian Olympiad 1995/2]
A chord $CD$ of a circle with center $O$ is perpendicular to a diameter $AB$. 
A chord $AE$ bisects the radius $OC$. Show that the line $DE$ bisects the chord $BC$.
\end{problem}
\begin{sol}
Let $M$ be the midpoint of $OC$ and let $DE$ intersect $BC$ at $N$. 
\begin{center}
\begin{asy}
real r = 2;
pair B, A, O;
A = (r, 0);
B = (-r, 0);
O = (A + B)/2;

pair G = B + (A - B)*0.65;
pair C, D;

C = (r**2 - G.x**2)**(1/2) * dir(90) + G;
D = (r**2 - G.x**2)**(1/2) * dir(-90) + G;

pair N, M;
N = (B + C)/2;
M = (O + C)/2;

pair E = extension(D, N, A, M);

D(CR(O, r));
D(A -- B -- C -- cycle);
D(B -- E -- A);
D(O -- C);
D(D -- C);
D(D -- E);
D(D -- A);
D(E -- C);

dot("$O$", O, align = S);
dot("$A$", A, align = SE);
dot("$B$", B, align = SW);
dot("$C$", C, align = NE);
dot("$D$", D, align = SE);
dot("$G$", G, align = SE);
dot("$E$", E, align = NW);
dot("$M$", M, align = SW);
dot("$N$", N, 1.5*dir(180));
\end{asy}
\end{center}
Let $\angle BAE = \alpha$ and $\angle EAC = \beta$.
Since $OM = MC$, 
\begin{align*}
             & \frac{MC}{OM} = \frac{AC \times \sin \beta}{OA \times \sin \alpha} \\
    \implies & \frac{AC \times \sin \beta}{OA \times \sin \alpha} = 1 \\
    \implies & \left(\frac{AC}{2\times OA}\right)\times \left(\frac{\sin \beta}{\sin \alpha}\right) = \frac{1}{2} \\
    \implies & \frac{\cos \angle BAC \times \sin \beta}{\sin \alpha} = \frac{1}{2} \\
    \implies & \boxed{\frac{\cos (\alpha+\beta) \times \sin \beta}{\sin \alpha} = \frac{1}{2}}
\end{align*}
If we can somehow show that 
\[
    \frac{EC \times \sin NEC}{EB \times \sin NEB} = 1
\]
then we'll be done. Now 
\[
    \angle NEB = \angle DEB = \angle DAB = \angle BAC = \alpha + \beta
\]
Since $DACE$ is cyclic
\[
    \angle DEC = 180\dg - \angle DAC = 180\dg - 2(\alpha + \beta) \implies \angle NEC = 180\dg - 2(\alpha + \beta)
\]
Using the sine law we get
\[
    \frac{EB}{\sin \alpha} = \frac{EC}{\sin \beta} = 2r \implies \frac{EC}{EB} = \frac{\sin \beta}{\sin \alpha}
\]
Hence
\begin{align*}
      \frac{EC \times \sin NEC}{EB \times \sin NEB}
    = &\left(\frac{EC}{EB}\right)\times \left(\frac{\sin (180\dg - 2\alpha - 2\beta)}{\sin (\alpha + \beta)}\right) \\
    = &\left(\frac{\sin \beta}{\sin \alpha}\right)\times \left(\frac{\sin {2(\alpha + \beta)}}{\sin (\alpha + \beta)}\right) \\
    = &\left(\frac{\sin \beta}{\sin \alpha}\right)\times 
       \left(\frac{2 \sin (\alpha + \beta) \cos (\alpha + \beta)}{\sin (\alpha + \beta)}\right) \\
    = & 2\times \left(\frac{\sin \beta \times \cos (\alpha + \beta)}{\sin \alpha}\right) = 1
\end{align*}
Therefore since 
\[
    \frac{NC}{BN} = \frac{EC \times \sin NEC}{EB \times \sin NEB} = 1 \implies NC = NB
\]
$N$ must be the midpoint of $BC$.
\end{sol}

\section{Stewart's Theorem}
\begin{theorem}[Stewart's Theorem]
    Let $\triangle ABC$ be a triangle and let $AX$ be a cevian of $\triangle ABC$. 
    If $BC = a, AC = b, AB = c, AX = x, BX = m$ and $CX = n$ then 
    \[
        b^{2}m + c^{2}n = a(x^{2} + mn)
    \]
\end{theorem}
\begin{proof}
    See \url{https://en.wikipedia.org/wiki/Stewart's_theorem}
\end{proof}

\end{document}
