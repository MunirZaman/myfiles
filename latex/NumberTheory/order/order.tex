\documentclass[11pt,numbers=noenddot,svgnames,dvipsnames]{scrartcl}
\usepackage[top=1in, left=1in, right=1in, bottom=1in]{geometry}
\usepackage[head, date]{munir}

\renewcommand{\pmod}[1]{\ (\mathrm{mod}\ #1)}

\title{Order of an Integer Modulo $n$}
\author{Munir Uz Zaman}
\date{Date: \today}

\begin{document}
\maketitle

\section{Orders}

\begin{definition}
    The order of an integer $a$ modulo $n$ where $a$ and $n$ are coprime integers 
    is the smallest positive integer $k$ such that $a^{k} \equiv 1 \pmod{n}$
\end{definition}
We will use the notation $\ord_{n} a$ to denote the order of $a$ modulo $n$. 
For example 2 has order 3 modulo 7. Therefore we can write  $\ord_{7} 2 = 3$. 

\begin{remark}
    If $\gcd(a, n) \neq 1$ then there does not exist any positive integer $k$ such that 
    $a^{k} \equiv 1 \pmod n$, because the linear congruence $ax \equiv 1 \pmod n$ does not 
    have a solution when $a$ and $n$ are not coprime. \\
    Therefore whenever we are talking about the order of $a$ modulo $n$, 
    it should be implicitly assumed that $a$ and $n$ are coprime integers.
\end{remark}

\begin{theorem}[Fundatmental Theorem of Orders]
    If $a$ is an integer then 
    \[
        a^{k} \equiv 1 \pmod{n} \iff \ord_{n} a \mid k
    \]
\end{theorem}
\begin{proof}
    TODO
\end{proof}

\begin{corollary}
    If $a$ is an integer then 
    \[
        \ord_{n} a \mid \phi(n)
    \]
\end{corollary}

\begin{theorem}
    If $p$ is a prime then there exists an $x$ such that 
    \[
        p \mid x^{2} + 1
    \]
    if and only if $p = 2$ or $p \equiv 1 \pmod{4}$
\end{theorem}
\begin{proof}
    We are going to first prove that if $p>2$ then 
    \[
        p \mid x^{2} + 1 \implies 4 \mid p - 1
    \]
    Now 
    \[
        x^{2} \equiv -1 \pmod p \implies x^{4} \equiv 1 \pmod p
    \]
    Therefore $\ord_{p} x \mid 4 \implies \ord_{p} x \in \{1, 2, 4\}$. 
    Clearly $\ord_{p}x$ is not $1$ or $2$ (why?). Thus $\ord_{p} x = 4$. 
    Hence
    \[
        \ord_{p} x \mid \phi(p) \implies 4 \mid p - 1
    \]
    Now we will prove the converse: if $p > 2$ and $p \equiv 1 \pmod 4$ then there exists 
    an $x$ such that $p \mid x^{2} + 1$. For this we take 
    \[
        x = \left(\frac{p - 1}{2}\right)!
    \]
    Now 
    \begin{align*}
        x & \equiv \left(\frac{p - 1}{2}\right)! \pmod p \\
          & \equiv \left(\frac{p - 1}{2}\right)\cdot \left(\frac{p - 2}{2}\right) \cdots 2 \cdot 1 \pmod p \\
          & \equiv \left(-\frac{p + 1}{2}\right)\cdot \left(-\frac{p + 2}{2}\right) \cdots -(p - 2) \cdot -(p - 1) \pmod p \\
          & \equiv (-1)^{\frac{p-1}{2}} \left(\frac{p + 1}{2}\right)\cdot \left(\frac{p + 2}{2}\right) \cdots (p - 2) \cdot (p - 1) \pmod p
    \end{align*}
    Therefore 
    \begin{align*}
        & x^{2} = \left(\frac{p - 1}{2}\right)! 
        \times (-1)^{\frac{p-1}{2}} \left(\frac{p + 1}{2}\right)\cdot \left(\frac{p + 2}{2}\right) \cdots (p - 2) \cdot (p - 1) \pmod p \\
        \implies & x^{2} = (-1)^{\frac{p - 1}{2}}(p - 1)! \pmod p
    \end{align*}
    Using Wilson's Theorem we have 
    \[
        x^{2} \equiv (-1)^{\frac{p-1}{2} + 1} \pmod p \implies x^{2} \equiv -1 \pmod p \implies p \mid x^{2} + 1
    \]
\end{proof}

\begin{lemma}[GCD Trick]
    If $a^{m} \equiv 1 \pmod N$ and $a^{n} \equiv 1 \pmod N$ then 
    \[
        a^{\gcd(m, n)} \equiv 1 \pmod N
    \]
\end{lemma}
\begin{proof}
    This is just the famous fact that $\gcd(a^{m} - 1, a^{n} - 1) = a^{\gcd(m, n)} - 1$ 
    phrased using modular arithmetic (how?). 
\end{proof}

\begin{lemma}
    If $p > 2$ is a prime and $p \mid x^{2^{n}} + 1$ where 
    $n$ is a positive integer, then $p \equiv 1 \pmod{2^{n + 1}}$.
\end{lemma}
\begin{proof}
    \[
        x^{2^{n}} \equiv -1 \pmod p \implies x^{2^{n + 1}} \equiv 1 \pmod p
    \]
    Therefore $\ord_{p} x \mid 2^{n + 1}$ which implies $\ord_{p} x = 2^{t}$ where 
    $1 \leq t \leq n + 1$. If $t < n + 1$ then 
    \[
        x^{2^{t}} \equiv 1 \pmod p \implies \left(x^{2^{t}}\right)^{2^{n - t}} \equiv x^{2^{n}} \equiv 1 \pmod p
    \]
    which contradicts our hypothesis. Therefore $t = n+1 \implies \ord_{p} x = 2^{n + 1}$. Hence 
    \[
        2^{n+1}\mid p-1 \implies p \equiv 1 \pmod{2^{n+1}}
    \]
\end{proof}

\begin{remark}
    Why can't $p = 2$? Because if $p = 2$ then $ 1 \equiv -1 \pmod 2$.
\end{remark}

\begin{example}
    Find all primes $p$ and $q$ such that $pq \mid 2^{p} + 2^{q}$.
\end{example}
\begin{sol}
    Let us assume that both $p$ and $q$ are coprime to 2.
    \begin{align*}
        pq \mid 2^p + 2^q &\implies 2^p + 2^q \equiv 0 \pmod p \\
                          &\implies 2 + 2^q \equiv 0 \pmod p \\
                          &\implies 2(2^{q-1} + 1) \equiv 0 \pmod p \\
                          &\implies 2^{q-1} \equiv -1 \pmod p 
    \end{align*}
    Likewise $2^{p-1} \equiv -1 \pmod q$. Suppose $k$ is the largest integer 
    such that $2^{k} \mid p-1$. That is, $2^{k}$ is the largest power of 2 that divides $p - 1$. 
    Let $p-1 = 2^{k}n$. 
    \[
        2^{p-1} \equiv 2^{2^{k}n} \equiv -1 \pmod q \implies 2^{k+1} \mid q - 1
    \]
    Therefore $q-1 = 2^{k+1}m$ where $m$ is an integer. Now 
    \[
        2^{q-1} \equiv 2^{2^{k+1}m} \equiv -1 \pmod p \implies 2^{k+2}\mid p - 1
    \]
    This contradicts our assumption that $2^{k}$ is the largest power of $2$ which 
    divides $p-1$. Hence both $p$ and $q$ cannot be coprime to 2. 

    Clearly $p=q=2$ is a valid solution. Now assume $q=2$ and $p>2$. 
    \begin{align*}
        2p \mid 2^{p} + 4 &\implies p \mid 2^{p-1} + 2 \\
                          &\implies 2^{p-1} + 2 \equiv 0 \pmod p \\
                          &\implies 3 \equiv 0 \pmod p \implies p = 3
    \end{align*}
    Therefore $(p,q) = (3, 2), (2, 3)$ are also a valid solutions. 

    Hence the solutions are $(2,2), (3,2), (2,3)$.
\end{sol}

\begin{example}
    Find all positive integers $m$ and $n$ such that 
    \[
        n \mid 1 + m^{3^{n}} + m^{2\cdot 3^{n}}
    \]
\end{example}
\begin{sol}
    \[
        n \mid 1 + m^{3^{n}} + m^{2\cdot 3^{n}}
        \implies n \mid m^{3^{n+1}} - 1
        \implies m^{3^{n+1}} \equiv 1 \pmod n
    \]
    Therefore $\ord_{n}(m) \mid 3^{n+1} \implies \ord_{n}(m) = 3^{t}$ 
    where $0\leq t \leq n+1$. 

    If $t = n+1$ then 
    \[
        \ord_{n}(m) \mid \phi(n) 
        \implies 3^{n+1} \mid \phi(n) 
        \implies 3^{n+1} \leq \phi(n) < n
    \]
    which is impossible. 

    If $t < n+1$, then 
    \[
        1 + m^{3^{n}} + m^{2\cdot 3^{n}} \equiv 0 \pmod n
        \implies 3 \equiv 0 \pmod n 
        \implies n \in \{1, 3\}
    \]
    If $n=1$, then
    \[
        1 \mid 1 + m^{3} + m^{6}
    \]
    which is true for all positive integer values of $m$. 
    If $n=3$, then 
    \[ m^{3^{t}} \equiv 1 \pmod 3\]
    which is only possible if $m \equiv 1 \pmod 3$. (Why?)

    Therefore the solutions are $(n, m) = (1, k), (3, 3k+1)$.
\end{sol}

\begin{example}
    Find all $n$ such that $n$ divides $2^{n} - 1$.
\end{example}
\begin{sol}
    Let $p$ be the smallest prime factor of $n$. Now 
    \begin{align*}
        \begin{matrix}
            & 2^n     \equiv 1 \pmod p \\
            & 2^{p-1} \equiv 1 \pmod p
        \end{matrix} \implies 2^{\gcd(p - 1, n)} \equiv 1 \pmod p
    \end{align*}
    Since $p$ is the smallest prime divisor of $n$ and $\gcd(p - 1, n) \mid n$, 
    we must have $\gcd(p - 1, n) = 1$ (why?). Hence 
    \[
        p \mid 2^1 - 1 \implies p \mid 1
    \]
    which is impossible. Therefore there does not exist such an $n$.
\end{sol}

\begin{problem}[China TST 2006]
    Find all positive integers $a$ and $n$ such that 
    \[
        (a+1)^{n}\equiv a^{n}\pmod n
    \]
\end{problem}
\begin{sol}
    We are gonna use the same idea as before. Let $p$ be smallest prime divisor of $n$. 
    \begin{align*}
                 (a+1)^n \equiv a^n \pmod p 
        \implies \left(1 + \frac{1}{a}\right)^n \equiv 1 \pmod p
    \end{align*}
    Now 
    \begin{equation*}
        \begin{matrix}
            \left(1 + \frac{1}{a}\right)^{n}   \equiv 1 \pmod p\\
            \left(1 + \frac{1}{a}\right)^{p-1} \equiv 1 \pmod p
        \end{matrix}
        \implies 
        \left(1 + \frac{1}{a}\right)^{\gcd(n, p-1)} \equiv 1\pmod p
    \end{equation*}
    Since $\gcd(n,p-1)\mid n$ and $\gcd(n,p-1) < p$, $\gcd(n,p-1)$ must be 1. 
    Therefore 
    \[
        \left(1 + \frac{1}{a}\right) \equiv 1 \pmod p 
        \implies a \equiv 1 \pmod p 
        \implies a \equiv 1 \pmod n
    \]
    Hence 
    \[
        2^{n} \equiv 1 \pmod n
    \]
    We've shown earlier that there does not exist any integer $n$ such that 
    \[
        n \mid 2^{n} - 1
    \]
    Therefore such integers do not exist.
\end{sol}

\begin{theorem}
    If $a$ is an integer such that $\ord_{n} a = k$, then 
    \[
        a^{i} \equiv a^{j} \pmod n \iff i \equiv j \pmod k
    \]
\end{theorem}
%TODO -> Proof

\begin{corollary}
    If $a$ has order $k$ modulo $n$, then the integers $a, a^{2}, \cdots, a^{k}$ 
    are incongruent modulo $n$.
\end{corollary}

\begin{example}
    Show that for all $n$, $3^{n} - 2^{n}$ is not divisible by $n$.
\end{example}
\begin{proof}
    For the sake of contradiction, assume $n\mid 3^{n} - 2^{n}$. Let $p$ be the smallest 
    prime divisor of $n$. 
    \[
        n \mid 3^{n} - 2^{n} \implies p \mid 3^{n} - 2^{n}
    \]
    Let $a$ be an integer such that $2a \equiv 1 \pmod p$. Now since $3^{n} \equiv 2^{n} \pmod p$ 
    \[
        2a \equiv 1 \pmod p \implies (2a)^{n} \equiv (3a)^{n} \equiv 1 \pmod p
    \]
    Therefore 
    \[
        \ord_{p}(3a) \mid n
    \]
    But since $\ord_{p}(3a) < p$ and $p$ is the smallest prime divisor of $n$, we must have 
    $\ord_{p}(3a) = 1$. Thus 
    \[
        3a \equiv 1 \pmod p \implies a \equiv 0 \pmod p
    \]
    This contradicts our assumption that $2a \equiv 1 \pmod p$. Hence such an integer $n$ cannot exist.
\end{proof}

\section{Primitive Roots}

\begin{definition}
    If the order of $g$ modulo $n$ is $\phi(n)$, then $g$ is called 
    a \vocab{primitive root} of $n$
\end{definition}
For example, 2 is a primitive root of 5. 
If $n$ is not a prime, then it is possible that $n$ does not have any primitive root. 
But for all prime there exists a primitive root.

\begin{theorem}
    If $p$ is a prime then the primitive root of $p$ exists. 
\end{theorem}
\begin{lemma}
    Given a primitive root $g$, each nonzero residue modulo $p$ can be 
    expressed uniquely as $g^{\alpha}$ where $\alpha \in \{1, 2, \cdots, p - 1\}$.
\end{lemma}

\begin{lemma}
    Let $p > 2$ be a prime. If $x$ is an integer, then 
    \begin{equation*}
        1^{x} + 2^{x} + \cdots + (p - 1)^{x} \equiv 
        \begin{cases}
            -1 &\text{if }(p - 1) \mid x \\
             0 &\text{otherwise}
        \end{cases}
        \pmod p
    \end{equation*}
\end{lemma}
\begin{proof}
    Let $g$ be a primitive root of $p$. 
    If $(p - 1)\mid x$ then 
    \[
        1^{x} + \cdots + (p - 1)^{x} 
        \equiv \sum_{\alpha = 1}^{p - 1} g^{\alpha x} 
        \equiv \sum_{\alpha = 1}^{p - 1} 1^{x} 
        \equiv (p - 1) \equiv -1 \pmod p
    \]
    If $(p - 1)\nmid x$ then 
    \[
        1^{x} + \cdots + (p - 1)^{x} 
        \equiv \sum_{\alpha = 1}^{p-1} g^{\alpha x} 
        \equiv g^{x}\left(\frac{g^{(p - 1)x} - 1}{g^{x} - 1}\right) 
        \equiv 0 \pmod p
    \]
\end{proof}

\end{document}
