\documentclass[11pt,numbers=noenddot,svgnames,dvipsnames]{scrartcl}
\usepackage[top=1in, left=1in, right=1in, bottom=1in]{geometry}
\usepackage[page, head, date, asy]{munir}
\title{EGMO : Chapter 3}
\author{Munir Uz Zaman}
\date{Date: \today}

\newcommand{\Homo}{\mathcal{H}}

\begin{document}
\maketitle

\section{Menelaus's and Ceva's Theorem}
\begin{theorem}[Menelaus's Theorem]
    Let $X, Y, Z$ be points on lines $BC, CA, AB$ in triangle $\triangle ABC$, 
    distinct from its vertices. Then $X, Y, Z$ are collinear if and only if 
    \[
        \frac{BX}{XC} \cdot \frac{CY}{YA} \cdot \frac{AZ}{ZB} = -1
    \]
    where the lengths are directed\footnote{Given collinear points $X, Y, Z$, we say that $\frac{XY}{YZ}$ is positive if $Y$ lies between $X$ and $Z$}.
\end{theorem}

\begin{theorem}[Ceva's Theorem]
    Let $X, Y, Z$ be points on lines $BC, CA, AB$ in triangle $\triangle ABC$, 
    distinct from its vertices. Then $X, Y, Z$ are concurrent if and only if 
    \[
        \frac{BX}{XC} \cdot \frac{CY}{YA} \cdot \frac{AZ}{ZB} = +1
    \]
    where the lengths are directed.
\end{theorem}

\section{Homothety}

\begin{definition}
    A \vocab{homothety} $\Homo$ is a transformation defined by a center $O$ and a real number $k$. 
    It sends a point $P$ to another point $\Homo(P)$, multiplying the distance from $O$ by $k$. 
    The number $k$ is called the scale factor.
\end{definition}

\begin{center}
\begin{asy}
size(3cm);
pair O = (0, 0);
pair P = (1, .5);
pair H = P*2;
D(O -- P -- H, dashed);
dot("$\mathcal{H}(P)$", H, align=N);
dot("$O$", O, align=S);
dot("$P$", P, align=NW);
\end{asy}
\end{center}
\end{document}
