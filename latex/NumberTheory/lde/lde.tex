\documentclass[11pt,numbers=noenddot,svgnames,dvipsnames]{scrartcl}
\usepackage[top=1in, left=1in, right=1in, bottom=1in]{geometry}
\usepackage{munir}

\title{Linear Diophantine Equations}
\author{Munir Uz Zaman}
\date{Date: \today}

\begin{document}
\maketitle

\section{Bezout's Identity}

\begin{theorem}[Bezout's Identity]
    If $a,b$ are non-zero integers and $d = \gcd(a, b)$ then there exists $x, y \in \mathbb{Z}$ such that 
    \[
        ax + by = d
    \]
\end{theorem}
\begin{proof}
    We will show that $d$ is the smallest integer in the set 
    \[
        S = \left\{ax + by > 0 \text{ and } x, y \in \mathbb{Z}\right\}
    \]
    By the well ordering principle, the set has a minimum element. Let $d$ be the minimum element of $S$. 
    We will first show that $d$ is a common divisor of $a$ and $b$. \\
    Suppose $a = qd + r$ where $0 \leq r < d$. Now 
    \begin{align*}
        a = qd + r \implies a = q(ax + by) + r \implies r = (1 - qx)a + (-qb)y
    \end{align*}
    If $r > 0$ then $r$ must be an element of $S$. But that contradicts our assumption that $d$ is the minimal element 
    of $S$ since $r < d$. Therefore $r = 0 \implies d \mid a$. Likewise we can show that $d \mid b$. \\
    Therefore $d$ is a common divisor of $a,b$. Now we need to show that $d$ is the largest common divisor of $d$. 
    Suppose $g$ is a common divisor of $a,b$ and $a = gm$ and $b = gn$. Now 
    \begin{align*}
        ax+ by = d \implies g(mx + ny) = d \implies g \mid d \implies g \leq d
    \end{align*}
    Thus $d$ is the largest common divisor of $a$ and $b$.
\end{proof}

\begin{corollary}
    If $a,b$ are coprime integers then there exists integers $x, y$ such that 
    \[
        ax + by = 1
    \]
\end{corollary}

\begin{theorem}[Euclid's Lemma]
    If $a \mid bc$ and $\gcd(a, b) = 1$ then $a \mid c$.
\end{theorem}
\begin{proof}
    Suppose $bc = ak$ where $k \in \mathbb{Z}$. 
    Since $\gcd(a, b) = 1$, there exists integers $x, y$ such that 
    \begin{align*}
        ax + by = 1 &\implies (ac)x + (bc)y = c \\
                    &\implies (ac)x + (ak)y = c \\
                    &\implies a(cx + ky) = c \\
                    &\implies a \mid c
    \end{align*}
\end{proof}

\begin{theorem}[General Bezout's Identity]
    If $a_{1}, a_{2}, \cdots, a_{n}$ are non-zero integers then there exists integers 
    $x_{1}, x_{2}, \cdots, x_{n}$ such that 
    \[
        a_{1}x_{1} + \cdots + a_{n}x_{n} = \gcd(a_{1}, \cdots, a_{n})
    \]
\end{theorem}

\end{document}
