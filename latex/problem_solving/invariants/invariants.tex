\documentclass[11pt,numbers=noenddot,svgnames,dvipsnames]{scrartcl}
\usepackage[top=1in, left=1in, right=1in, bottom=1in]{geometry}
\usepackage[page, head]{munir}

\title{Invariants}
\author{Munir Uz Zaman}
\date{\today}

\begin{document}
\maketitle

An \emph{invariant} is a quality or quantity that never changes. 
For example, the net energy of a closed system is invariant 
(This is known as \emph{the law of conservation of energy}). 
A \emph{monovariant} or \emph{semi-invariant} is a quantity 
that either always increases or always decreases. \\
Finding an invariant is a common idea in problems asking to prove that 
something cannot be achieved or some state cannot be reached (by applying some algorithm). 
Monovariants are also very efficient in showing that the corresponding process must stop after finitely many moves.

\section{Examples}

\begin{problem}
    You are given the set of integers $1, 2, \cdots, n$. 
    In each step you can pick any two numbers $a$ and $b$, 
    remove those two numbers and add $ab+a+b$ to the set. 
    Show that after $n-1$ steps the number that will remain will always be $(n+1)! -1$.
\end{problem}
\begin{sol}
    Let $x_{1}, x_{2}, \cdots, x_{k}$ be the numbers currently in the set. 
    Now let us consider the value, $X = (x_{1}+1)(x_{2}+1)\cdots (x_{k}+1)$. 
    After removing any two numbers $x_{i}$ and $x_{j}$, and adding 
    $x_{i}x_{j} + x_{i} + x_{j}$ to the set, the value $X$ will still be the same.
    \[
        X = (x_{1} + 1) \cdots (1 + x_{i} + x_{j} + x_{i}x_{j}) \cdots (x_{k} + 1) 
          = (x_{1} + 1) \cdots (x_{i} + 1) (x_{j} + 1) \cdots (x_{k} + 1)
    \]
    Therefore $X$ is an invariant. Initially $X=(n+1)!$. If $A$ is the integer 
    that will remain after $n-1$ steps then, 
    \[
        X = A+1 \implies A = (n+1)! - 1
    \]
    We are done!
\end{sol}

\begin{problem}
    The integers $x_{1}, x_{2}, \cdots, x_{n}$ are all written around a circle. 
    You can pick any 4 successive integers $x_{k}, x_{k+1}, x_{k+2}, x_{k+3}$. 
    If $(x_{k} - x_{k+3})(x_{k+1} - x_{k+2}) < 0$ then 
    you can replace $x_{k+1}$ with $x_{k+2}$ and $x_{k+2}$ with $x_{k+1}$ (swap $x_{k+1},x_{k+2}$). 
    Show that you cannot do this indefinitely.
\end{problem}
\begin{sol}
    It makes sense to try to find a monovariant as the problem wants us to 
    show that a certain process cannot go on indefinitely. Let us consider 
    the sum, 
    (assume that if $k > n$ then $x_{k} = x_{r}$ where $r = k - \floor{\frac{k}{n}}n$)
    \[
        S = x_{1}x_{2} + x_{2}x_{3} + \cdots + x_{n-1}x_{n} + x_{n}x_{1}
    \]
    Now if $(x_{k} - x_{k+3})(x_{k+1} - x_{k+2}) < 0$ then we can replace 
    $x_{k+1}$ with $x_{k+2}$ and $x_{k+2}$ with $x_{k+1}$. Suppose $S_{1}$ 
    is the new value of the sum after swapping $x_{k+1}$ and $x_{k+2}$, and $S_{0}$ 
    is the old value of the sum.
    \[
        S_{0} = x_{1}x_{2} + \cdots + x_{k}x_{k+1} + x_{k+1}x_{k+2} + x_{k+2}x_{k+3} + \cdots + x_{n}x_{1}
    \]
    \[
        S_{1} = x_{1}x_{2} + \cdots + x_{k}x_{k+2} + x_{k+2}x_{k+1} + x_{k+1}x_{k+3} + \cdots + x_{n}x_{1}
    \]
    The difference between the two values will be 
    \[
        S_{1} - S_{0} = x_{k}x_{k+2} - x_{k}x_{k+1} + x_{k+1}x_{k+3} - x_{k+2}x_{k+3} 
        \implies S_{1} - S_{0} = (x_{k} - x_{k+3})(x_{k+2} - x_{k+1})
    \]
    Since $(x_{k} - x_{k+3})(x_{k+1} - x_{k+2}) < 0 \implies (x_{k} - x_{k+3})(x_{k+2} - x_{k+1}) > 0$, 
    $S_{1} > S_{0}$. Therefore the value of the sum $S$ is strictly increasing in each step. 
    Now we just need to show that $S$ is bounded above, that is, we need to find an upper bound 
    for the sum. Let us assume $X = \max(x_{1}, \cdots, x_{n})$. 
    For any $i$, $x_{i}x_{i+1} \leq X^{2} \implies x_{i}x_{i+1} < X^{2} + 1$. Thus,
    \[
        S = x_{1}x_{2} + \cdots + x_{n}x_{1} < n \times (X^{2} + 1)
    \]
    And so the value of $S$ cannot increase forever which implies that the process cannot go on indefinitely.
\end{sol}
% TODO: add some examples
\begin{proposition}
    Let $a_{n}$ be a sequence recursively defined as,
    \[
        a_{n} = \sum_{i=1}^{k} n_{i} a_{n-i} = n_{1} a_{n-1} + \cdots + n_{k} a_{n-k}
    \]
    If $N$ is an integer such that, $\sum_{i=1}^{k} n_{i} \equiv 1 \pmod{N}$ and 
    $X_{n}$ is a sequence such that,
    \[
        X_{n} = \sum_{i=1}^{k} x_{i} a_{n+i-1} = x_{1} a_{n} + \cdots + x_{k} a_{n+k-1}
    \]
    where, $x_{i} \equiv \sum_{j=1}^{i} n_{k-j+1} \pmod{N}$ then the sequence 
    $X_{n}$ is invariant modulo $N$, that is, $X_{n} \equiv X_{n-1} \pmod{N}$ 
    for all $n$.
\end{proposition}
\begin{proof}
    We first prove that, $x_{i+1} \equiv x_{i} + n_{k-i} \pmod{N}$. 
    \begin{align*}
        x_{i+1} \equiv \sum_{j=1}^{i+1} n_{k-j+1} 
                \equiv \sum_{j=1}^{i} n_{k-j+1} + n_{k-i} 
                \equiv x_{i} + n_{k-i} \pmod{N}
    \end{align*}
    Since $\sum_{i=1}^{k} n_{i} \equiv 1 \pmod{N} \implies x_{k} \equiv 1 \pmod{N}$,
    \begin{align*}
                 X_{n} &\equiv \sum_{i=1}^{k} x_{i} a_{n+i-1} \pmod{N} \\
        \implies X_{n} &\equiv \sum_{i=1}^{k-1} x_{i} a_{n+i-1} + x_{k} a_{n+k-1} \pmod{N} \\
        \implies X_{n} &\equiv \sum_{i=1}^{k-1} x_{i} a_{n+i-1} + a_{n+k-1} \pmod{N}
    \end{align*}
    Now since $a_{n+k-1}\equiv \sum_{i=1}^{k} n_{i} a_{n+k-i-1}$,
    \begin{align*}
       \implies X_{n} &\equiv \sum_{i=1}^{k-1} x_{i} a_{n+i-1} + a_{n+k-1} \pmod{N} \\
       \implies X_{n} &\equiv \sum_{i=1}^{k-1} x_{i} a_{n+i-1} + \sum_{i=1}^{k} n_{i} a_{n+k-i-1} \pmod{N}
    \end{align*}
    Since $\sum_{i=1}^{k} n_{i} a_{n+k-i-1} = \sum_{i=0}^{k-1} n_{k-i} a_{n+i-1}$,
    \begin{align*}
       \implies X_{n} &\equiv \sum_{i=1}^{k-1} x_{i} a_{n+i-1} + \sum_{i=1}^{k} n_{i} a_{n+k-i-1} \pmod{N} \\
       \implies X_{n} &\equiv \sum_{i=1}^{k-1} x_{i} a_{n+i-1} + \sum_{i=0}^{k-1} n_{k-i} a_{n+i-1} \pmod{N} \\
       \implies X_{n} &\equiv n_{k} a_{n-1} + \sum_{i=1}^{k-1} \left( x_{i} + n_{k-i} \right) a_{n+i-1} \pmod{N} \\
       \implies X_{n} &\equiv x_{1} a_{n-1} + \sum_{i=1}^{k-1} x_{i+1} a_{n+i-1} \pmod{N} \\
       \implies X_{n} &\equiv x_{1} a_{n-1} + \sum_{i=2}^{k} x_{i} a_{n+i-2} \pmod{N} \\
       \implies X_{n} &\equiv \sum_{i=1}^{k} x_{i} a_{n-1+i-1} \pmod{N} \\
       \implies X_{n} &\equiv X_{n-1} \pmod{N}
    \end{align*}
    And we are done!
\end{proof}
\end{document}

