\documentclass[11pt,numbers=noenddot,svgnames,dvipsnames]{scrartcl}
\usepackage[top=1in, left=1in, right=1in, bottom=1in]{geometry}
\usepackage[head, date]{munir}

\renewcommand{\pmod}[1]{\ (\mathrm{mod}\ #1)}

\title{Order of an Integer Modulo $n$}
\author{Munir Uz Zaman}
\date{Date: \today}

\begin{document}
\maketitle

\section{Orders}

\begin{definition}
    The order of an integer $a$ modulo $n$ where $a$ and $n$ are coprime integers 
    is the smallest positive integer $k$ such that $a^{k} \equiv 1 \pmod{n}$
\end{definition}
We will use the notation $\ord_{n} a$ to denote the order of $a$ modulo $n$. 
For example 2 has order 3 modulo 7. Therefore we can write  $\ord_{7} 2 = 3$. 

\begin{remark}
    If $\gcd(a, n) \neq 1$ then there does not exist any positive integer $k$ such that 
    $a^{k} \equiv 1 \pmod n$, because the linear congruence $ax \equiv 1 \pmod n$ does not 
    have a solution when $a$ and $n$ are not coprime. \\
    Therefore whenever we are talking about the order of $a$ modulo $n$, 
    it should be implicitly assumed that $a$ and $n$ are coprime integers.
\end{remark}

\begin{theorem}[Fundatmental Theorem of Orders]
    If $a$ is an integer then 
    \[
        a^{k} \equiv 1 \pmod{n} \iff \ord_{n} a \mid k
    \]
\end{theorem}
\begin{proof}
    TODO
\end{proof}

\begin{corollary}
    If $a$ is an integer then 
    \[
        \ord_{n} a \mid \phi(n)
    \]
\end{corollary}

\begin{theorem}
    If $p$ is a prime then there exists an $x$ such that 
    \[
        p \mid x^{2} + 1
    \]
    if and only if $p = 2$ or $p \equiv 1 \pmod{4}$
\end{theorem}
\begin{proof}
    We are going to first prove that if $p>2$ then 
    \[
        p \mid x^{2} + 1 \implies 4 \mid p - 1
    \]
    Now 
    \[
        x^{2} \equiv -1 \pmod p \implies x^{4} \equiv 1 \pmod p
    \]
    Therefore $\ord_{p} x \mid 4 \implies \ord_{p} x \in \{1, 2, 4\}$. 
    Clearly $\ord_{p}x$ is not $1$ or $2$ (why?). Thus $\ord_{p} x = 4$. 
    Hence
    \[
        \ord_{p} x \mid \phi(p) \implies 4 \mid p - 1
    \]
    Now we will prove the converse: if $p > 2$ and $p \equiv 1 \pmod 4$ then there exists 
    an $x$ such that $p \mid x^{2} + 1$. For this we take 
    \[
        x = \left(\frac{p - 1}{2}\right)!
    \]
    Now 
    \begin{align*}
        x & \equiv \left(\frac{p - 1}{2}\right)! \pmod p \\
          & \equiv \left(\frac{p - 1}{2}\right)\cdot \left(\frac{p - 2}{2}\right) \cdots 2 \cdot 1 \pmod p \\
          & \equiv \left(-\frac{p + 1}{2}\right)\cdot \left(-\frac{p + 2}{2}\right) \cdots -(p - 2) \cdot -(p - 1) \pmod p \\
          & \equiv (-1)^{\frac{p-1}{2}} \left(\frac{p + 1}{2}\right)\cdot \left(\frac{p + 2}{2}\right) \cdots (p - 2) \cdot (p - 1) \pmod p
    \end{align*}
    Therefore 
    \begin{align*}
        & x^{2} = \left(\frac{p - 1}{2}\right)! 
        \times (-1)^{\frac{p-1}{2}} \left(\frac{p + 1}{2}\right)\cdot \left(\frac{p + 2}{2}\right) \cdots (p - 2) \cdot (p - 1) \pmod p \\
        \implies & x^{2} = (-1)^{\frac{p - 1}{2}}(p - 1)! \pmod p
    \end{align*}
    Using Wilson's Theorem we have 
    \[
        x^{2} \equiv (-1)^{\frac{p-1}{2} + 1} \pmod p \implies x^{2} \equiv -1 \pmod p \implies p \mid x^{2} + 1
    \]
\end{proof}

\begin{lemma}[GCD Trick]
    If $a^{m} \equiv 1 \pmod N$ and $a^{n} \equiv 1 \pmod N$ then 
    \[
        a^{\gcd(m, n)} \equiv 1 \pmod N
    \]
\end{lemma}
\begin{proof}
    This is just the famous fact that $\gcd(a^{m} - 1, a^{n} - 1) = a^{\gcd(m, n)} - 1$ 
    phrased using modular arithmetic (how?). 
\end{proof}

\begin{example}
    Find all $n$ such that $n$ divides $2^{n} - 1$.
\end{example}
\begin{sol}
    Let $p$ be the smallest prime factor of $n$. Now 
    \begin{align*}
        \begin{matrix}
            & 2^n     \equiv 1 \pmod p \\
            & 2^{p-1} \equiv 1 \pmod p
        \end{matrix} \implies 2^{\gcd(p - 1, n)} \equiv 1 \pmod p
    \end{align*}
    Since $p$ is the smallest prime divisor of $n$ and $\gcd(p - 1, n) \mid n$, 
    we must have $\gcd(p - 1, n) = 1$ (why?). Hence 
    \[
        p \mid 2^1 - 1 \implies p \mid 1
    \]
    which is impossible. Therefore there does not exist such an $n$.
\end{sol}

\begin{theorem}
    If $a$ is an integer such that $\ord_{n} a = k$, then 
    \[
        a^{i} \equiv a^{j} \pmod n \iff i \equiv j \pmod k
    \]
\end{theorem}
%TODO -> Proof

\begin{corollary}
    If $a$ has order $k$ modulo $n$, then the integers $a, a^{2}, \cdots, a^{k}$ 
    are incongruent modulo $n$.
\end{corollary}

\section{Primitive Roots}

\begin{definition}
    If the order of $g$ modulo $n$ is $\phi(n)$, then $g$ is called 
    a \vocab{primitive root} of $n$
\end{definition}
If $n$ is not a prime, then it is possible that $n$ does not have any primitive root. 
But for all prime there exists a primitive root.

\begin{theorem}
    If $p$ is a prime then the primitive root of $p$ exists. 
\end{theorem}
\begin{lemma}
    Given a primitive root $g$, each nonzero residue modulo $p$ can be 
    expressed uniquely as $g^{\alpha}$ where $\alpha \in \{1, 2, \cdots, p - 1\}$.
\end{lemma}

\end{document}
