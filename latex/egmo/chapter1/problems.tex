\documentclass[11pt,numbers=noenddot,svgnames,dvipsnames]{scrartcl}
\usepackage[top=1in, left=1in, right=1in, bottom=1in]{geometry}
\usepackage[page, head, date, asy]{munir}

\title{EGMO:Chpater 1}
\author{Munir Uz Zaman}
\date{Date: \today}

\begin{document}
\maketitle

\begin{problem}[JMO 2011/5]
    Points $A, B, C, D, E$ lie on a circle $\omega$ and point $P$ lies outside the 
    circle. The given points are such that 
    \begin{itemize}
        \ii lines $PB$ and $PD$ are tangent to $\omega$.
        \ii $P, A, C$ are collinear.
        \ii $DE \parallel AC$.
    \end{itemize}
    Show that $BE$ bisects $AC$.
\end{problem}

\begin{problem}[Lemma 1.44 Three Tangents]
    Let $ABC$ be an acute angled triangle. Let $BE$ and $CF$ be the altitudes of 
    $\triangle ABC$ and let $M$ be the midpoint of $BC$. Prove that $ME$, $MF$ and 
    the line through $A$ parallel to $BC$ are all tangents to the circle $AEF$.
\end{problem}
\begin{center}    
\begin{asy}
pair A, B, C;
A = (3, 5);
B = (0, 0);
C = (6, 0);

pair M = (B + C)/2;
pair E, F;
E = foot(B, A, C);
F = foot(C, A, B);

D(rightanglemark(B, E, C));
D(rightanglemark(C, F, B));

D(A--B--C--cycle);

D(B -- E);
D(C -- F);
D(E -- F);
D(M -- E);
D(M -- F);

pair H = orthocenter(A, B, C);
D(H -- A);

pair X, Y;
X = A + (B - C)/2;
Y = A + (C - B)/2;
D(X -- Y, Arrows());

D(circumcircle(B, E, C), dashed+red);
D(circumcircle(A, E, F), opacity(.5));

dot("$A$", A, align=N);
dot("$B$", B, align=SW);
dot("$C$", C, align=SE);
dot("$M$", M, align=S);
dot("$E$", E, align=NE);
dot("$F$", F, align=NW);
dot("$X$", X, align=S);
dot("$Y$", Y, align=S);
label("$\ell$", X + 3*(Y - X)/4, align=N);
\end{asy}
\end{center}
\begin{proof}
    Since $\triangle BEC$ is a right triangle and $M$ is the midpoint of the hypotenuse $BC$, we have $MB = ME = MC$. 
    Likewise $MB = MF = MC$. Therefore 
    \[
        MB = ME = MF = MC
    \]
    Hence $B, E, F, C$ are all lying on the same circle with center $M$. If we can show that $\angle FAE = \angle FEM$ then 
    it'll be shown that $EM$ is tangent to circle $AFE$. 
    \begin{align*}
        \angle FEM = \angle FEB + \angle MEB 
    \end{align*}
    Now since $ME = MB$, 
    \begin{align*}
                 & \angle FEM = \angle FEB + \angle MEB \\
        \implies & \angle FEM = \angle FEB + \angle MBE 
    \end{align*}
    Since $CBEF$ is cyclic, $\angle FEB =  \angle BCF$.
    \begin{align*}
                 & \angle FEM = \angle FEB + \angle MBE \\
        \implies & \angle FEM = \angle BCF + \angle MBE \\
        \implies & \angle FEM = 180\dg - \angle BHC \\
        \implies & \angle FEM = 180\dg - \angle FHE \\ 
        \implies & \angle FEM = \angle FAE
    \end{align*}
    Therefore $EM$ is tangent to circle $AFE$. Likewise $FM$ is also tangent to $AFE$. Now we need to show 
    that $\ell$ is tangent to circle $AFE$. It suffices to show that $\angle XAB = \angle AEF$. \\
    \begin{align*}
                 & \angle AEF = \angle EFC + \angle ECF \\
        \implies & \angle AEF = 180\dg - \angle FEC \\
        \implies & \angle AEF = \angle CBF \\
        \implies & \angle AEF = \angle XAB
    \end{align*}
\end{proof}

\begin{problem}[Canada 1997/4]
    The point $O$ is inside the parallelogram $ABCD$ such that $\angle AOB + \angle COD = 180\dg$. 
    Show that $\angle OBC = \angle ODC$.
\end{problem}
\begin{center}
\begin{asy}
pair A, B, C, D, O;
A = (1, 3);
B = (0, 0);

pair v = (5, 0);
C = B + v;
D = A + v;

real alpha = degrees(atan2(A.y, A.x));
pair X, Y;
real theta = 20;
pair X1 = B + dir(theta);
pair X2 = D + dir(-(180 - alpha + theta));
pair O = extension(B, X1, D, X2);

D(B -- O);
D(D -- O);
D(A -- O);
D(C -- O);

pair Q = O - v; // o prime
D(A -- Q);
D(B -- Q);
D(Q -- O);
D(circumcircle(A, Q, B), red+dashed);

//dot(X1);dot(X2);
D(A -- B -- C -- D -- cycle);

D(anglemark(Q, O, B, t=12), red);
D(anglemark(Q, A, B, t=12), red);
D(anglemark(C, B, O, t=12), red);
D(anglemark(O, D, C, t=12), red);

dot("$A$", A, align=NW);
dot("$B$", B, align=SW);
dot("$C$", C, align=SE);
dot("$D$", D, align=NE);
dot("$O'$", Q, align=W);
dot("$O$", O, align=E);
\end{asy}
\end{center}
\begin{proof}
    Apply the translation $T \colon \overrightarrow{P} \to \overrightarrow{P} + \overrightarrow{CB}$ on points $C, O, D$. 
    Let $T(O) = O'$. Clearly $A, B$ are the points $D, C$ after applying the translation $T$. 
    \[
        \angle COD + \angle AOB = 180\dg \implies \angle BO'A + \angle AOB = 180\dg
    \]
    Therefore $AO'BO$ is cyclic. Hence $\angle O'AB = \angle O'OB$. As $OO' \parallel BC$, 
    \[
        \angle O'AB = \angle O'OB = \angle OBC \implies \angle ODC = \angle OBC
    \]
\end{proof}

\begin{problem}[Simson line]
    Let $ABC$ be a triangle and let $P$ be any point on the circumcircle of $\triangle ABC$. 
    Let $X, Y, Z$ be the feet of the perpendiculars from $P$ onto lines $BC, CA, AB$. 
    Prove that $X, Y, Z$ are collinear.
\end{problem}
\begin{center}
\begin{asy}
pair A, B, C;
real r = 2;

A = r*dir(100);
B = r*dir(180 + 30);
C = r*dir(-20);
pair P = r*dir(70);

D(A -- B -- C -- cycle);
D(circumcircle(A, B, C));

pair X, Y, Z;
X = foot(P, B, C);
Y = foot(P, C, A);
Z = foot(P, A, B);

D(P -- X);
D(P -- Y);
D(P -- Z);
D(P -- A);
D(P -- C);
D(A -- Z);
D(X -- Y -- Z);

D(rightanglemark(P, X, C, s = 4));
D(rightanglemark(P, Y, C, s = 4));
D(rightanglemark(P, Z, A, s = 4));

D(circumcircle(P, X, Y), dashed+red);
D(circumcircle(P, Y, Z), dashed+red);

dot("$A$", A, align=NW);
dot("$B$", B, align=SW);
dot("$C$", C, align=SE);
dot("$P$", P, align=NE);

dot("$X$", X, align=S);
dot("$Y$", Y, align=SW);
dot("$Z$", Z, align=NW);
\end{asy}
\end{center}
\begin{proof}
    It suffices to show that $\angle XYC = \angle AYZ$. Since $XYPC$ is cyclic, $\angle XYC = \angle XPC$. 
    \begin{align*}
        \angle XYC = \angle XPC = 90\dg - \angle PCX
    \end{align*}
    Since $ABCP$ is cyclic, $\angle PCX = 180\dg - \angle PAB$. Therefore 
    \begin{align*}
        \angle XYC & = 90\dg - \angle PCX \\
                   & = \angle PAB - 90\dg \\
                   & = \angle ZPA
    \end{align*}
    Since $AZPY$ is cyclic, $\angle ZPA = \angle AYZ$. Hence 
    \begin{align*}
        \angle XYZ = \angle AYZ
    \end{align*}
\end{proof}
\end{document}
