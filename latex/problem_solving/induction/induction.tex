\documentclass[11pt,numbers=noenddot,svgnames,dvipsnames]{scrartcl}
\usepackage[top=1in, left=1in, right=1in, bottom=1in]{geometry}
\usepackage[page, head, date]{munir}
\usepackage{mathtools}
\usepackage{microtype}

\title{Mathematical Induction}
\author{Munir Uz Zaman}
\date{Date: \today}

\begin{document}
\maketitle
\tableofcontents

\section{Preliminaries}
\begin{definition}
A \vocab{proposition} in mathematics is a statement that is either true or false.
\end{definition}
For example, ``2 + 2 = 4'' and ``19 is a prime number'' both are true mathematical statements. 
Here are some more examples of propositions.
\begin{proposition}
    If $f(n) = n^{2} + n + 41$ then $f(n)$ is a prime number for all non-negative integers $n$.
\end{proposition}
This is a proposition but the proposition is not true for all non-negative integers. 
For example, if $n=40$ then 
\[
    f(40) = 40^{2} + 40 + 41 = 40^{2} + 2\times 40 + 1 = 41^{2}
\]
\begin{proposition}[Goldbach's Conjecture]
    Every integer greater than 2 is a sum of two primes.
\end{proposition}
Goldbach's Conjecture is also a proposition but so far no one has been able to 
prove that it is true.

\begin{definition}
    A \vocab{predicate} is a proposition whose truth depends on one or more variables.
\end{definition}
For example, ``$n$ is a prime number'' is a predicate as its truth depends on the value of $n$. 
For $n=3$ the statement is true but for $n=12$ the statement is false. 
A function-like notation is used to denote a predicate supplied with specific variable values. 
For example, we might use the name ``P'' for the predicate above:
\[
P(n) : n\text{ is a prime number}
\]
Like before, we can say that $P(3)$ is true and $P(12)$ is false.

\begin{definition}
    An \vocab{axiom} is a proposition which is accepted as true without any proof.
\end{definition}
For example, ``$a = b \iff a + c = b + c$'' and ``two sets are equal if and only if they have the same elements'' are 
examples of axioms.

\section{The Induction Principle}

\begin{center}
{\itshape
Mathematical induction proves that we can climb as high
as we like on a ladder, by proving that we can climb onto
the bottom rung (the basis) and that from each rung we
can climb up to the next one (the step).}
\end{center}
\begin{flushright}
\textsf{- Concrete Mathematics}
\end{flushright}

The \vocab{induction principle} claims that if $\mathcal{P}(n)$ is some predicate and if
\begin{itemize}
        \ii $\mathcal{P}(n_{0})$ is true where $n_{0}$ is some integer and 
        \ii $\mathcal{P}(k) \implies \mathcal{P}(k+1)$ where $k\geq n_{0}$ is an integer
\end{itemize}
then $\mathcal{P}(n)$ is true for all integers $n \geq n_{0}$. We will later prove the induction principle but first 
let's take a look at some examples.

\begin{example}
    Show that for all $n \geq 1$
    \[
        1 + 2 + \cdots + n = \frac{n(n+1)}{2}
    \]
\end{example}
\begin{proof}
We have the predicate 
\[
    P(n) \colon 1 + 2 + \cdots + n = \frac{n(n+1)}{2} 
\]
We want show that $P(n)$ is true for all $n\geq 1$. Clearly $P(1)$ is true. Now we just need to show that 
$P(k) \implies P(k+1)$. Suppose $P(k)$ is true where $k \geq 1$. Now
\begin{align*}
             & 1 + 2 + \cdots + k = \frac{k(k+1)}{2} \\
    \implies & 1 + 2 + \cdots + k + (k+1) = (k+1) + \frac{k(k+1)}{2} \\
    \implies & 1 + 2 + \cdots + (k+1) = (k+1) \left(1 + \frac{k}{2}\right) \\
    \implies & 1 + 2 + \cdots + (k+1) = \frac{(k+1)(k + 2)}{2}
\end{align*}
And that's it! We just proved that if $P(k)$ is true then $P(k+1)$ is also true. Now from the induction 
principle, we can say that $P(n)$ is true for all $n \geq 1$. 
\end{proof}

There are two main steps in an inductive proof. First we show that $\mathcal{P}(n_{0})$ is 
true where $n_{0}$ is an integer. This step is known as the \vocab{base step} or the \vocab{induction basis}. 
Next we prove that if $k \geq n_{0}$ is an integer and $\mathcal{P}(k)$ is true then 
$\mathcal{P}(k+1)$ is also true. This step is called the \vocab{inductive step}. 
To prove the inductive step we assume that $\mathcal{P}(k)$ is true and then use 
this assumption to show that $\mathcal{P}(k + 1)$ must also be true. The hypothesis that 
$\mathcal{P}(k)$ is true for some integer $k\geq n_{0}$ is called the \vocab{induction hypothesis}. 
Here's another example of proof by induction.

\begin{example}
    Show that for all $n \in \mathbb{N}$
    \[
        1^{3} + 2^{3} + \cdots + n^{3} = (1 + 2 + \cdots + n)^{2}
    \]
\end{example}
\begin{proof}
    Earlier we proved that 
    \[
        1 + 2 + \cdots + n = \frac{n(n + 1)}{2}
    \]
    Therefore it suffices to show that 
    \[
        1^{3} + 2^{3} +  \cdots + n^{3} = \frac{n^{2}(n+1)^{2}}{4}
    \]
    for all $n \geq 1$. For $n=1$ the statement is clearly true. We can prove this just by plugging in $n = 1$ and 
    then showing that both sides are equal.
    \[
        1^{3} = \frac{1^{2}(1 + 1)^{2}}{4} \implies 1 = 1
    \]
    Now we need to show that if the statement holds true for some integer $k\geq 1$ then it must also hold true for $k + 1$. 
    Suppose the statement is true for $n = k$ for some integer $k\geq 1$.
    \begin{align*}
                 & 1^{3} + 2^{3} + \cdots + k^{3} = \frac{k^{2}(k + 1)^{2}}{4} \\
        \implies & 1^{3} + 2^{3} + \cdots + k^{3} + (k + 1)^{3} = \frac{k^{2}(k + 1)^{2}}{4} + (k + 1)^{3} \\
        \implies & 1^{3} + 2^{3} + \cdots + k^{3} + (k + 1)^{3} = \frac{k^{2}(k + 1)^{2} + 4(k + 1)^{3}}{4} \\
        \implies & 1^{3} + 2^{3} + \cdots + k^{3} + (k + 1)^{3} = \frac{(k + 1)^{2}(k^{2} + 4k + 4)}{4} \\
        \implies & 1^{3} + 2^{3} + \cdots + k^{3} + (k + 1)^{3} = \frac{(k + 1)^{2}(k + 2)^{2}}{4}
    \end{align*}
    Now from the induction principle, we can say that the statement is true for all $n \geq 1$.
\end{proof}

Okay, enough with examples. We will now prove the induction principle!

\begin{theorem}[Induction Principle]
    If $\mathcal{P}(n)$ is some predicate and if
    \begin{itemize}
            \ii $\mathcal{P}(n_{0})$ is true where $n_{0}$ is some integer and 
            \ii $\mathcal{P}(k) \implies \mathcal{P}(k+1)$ where $k\geq n_{0}$ is an integer
    \end{itemize}
    then $\mathcal{P}(n)$ is true for all integers $n \geq n_{0}$.
\end{theorem}
\begin{proof}
    Let $Q(n)$ be the predicate $\mathcal{P}(n_{0} + n)$.
    \[
        Q(n) : \mathcal{P}(n_{0} + n)
    \]
    We need to show that $Q(n)$ is  true for all $n \in \mathbb{N}$ if
    \begin{itemize}
        \ii $Q(0)$ is true and 
        \ii $Q(n) \implies Q(n + 1)$ for all $n \in \mathbb{N}$.
    \end{itemize}
    Let $T$ be the set of all non-negative integers for which $Q(n)$ is true and let 
    $F$ be the set of all non-negative integers for which $Q(n)$ is false. It suffices to 
    show that $F$ is an empty set. \\
    For the sake of contradiction, let us assume that $F$ is non-empty. Since $F$ is a non-empty 
    set of non-negative integers, there must exist a minimal element of $F$. Let $k$ be the smallest 
    element of $F$. Since $k > 0\implies k-1 \geq 0$ and $k - 1 \not \in F$, $k - 1$ must be an element 
    of $T$. Since $Q(k - 1)\implies Q(k)$, $k$ must also be an element of $T$. But that contradicts our 
    assumption that $k \in F$ as $Q(k)$ cannot be both true and false. Therefore $F$ does not have a minimal 
    element which implies $F$ must be an empty set. 
\end{proof}

\begin{exercise}
Prove using induction that for all $n\in \mathbb{N}$
\[
    1^{2} + 2^{2} + \cdots + n^{2} = \frac{n(n + 1)(2n + 1)}{6}
\]
\end{exercise}
\begin{exercise}
Prove using induction that if $r$ is a real number not equal to 1 then for all $n \in \mathbb{N}$
\[
    1 + r + r^{2} + \cdots + r^{n} = \frac{r^{n + 1} - 1}{r - 1}
\]
\end{exercise}
\begin{exercise}
    Show that if $x, y$ are real numbers and if $n \geq 2$ is a positive integer then 
    \[
        x^{n} - y^{n} = (x - y)\left(x^{n-1} + x^{n-2}y + \cdots + xy^{n-2} + y^{n-1} \right)
    \]
\end{exercise}

\begin{example}
    Show that for all $n\in \mathbb{N}$
    \[
        \frac{1}{1\times 2} + \frac{1}{2 \times 3} + \cdots + \frac{1}{n(n+1)} = \frac{n}{n + 1}
    \]
\end{example}
\begin{proof}
    For $n = 1$ the statement is true.
    \[
        \frac{1}{1 \times 2} = \frac{1}{1 + 1} \implies \frac{1}{2} = \frac{1}{2}
    \]
    Assume the statement is true for some integer $k\geq 1$. Now we prove that the 
    statement must also be true for $k + 1$.
    \begin{align*}
                 & \frac{1}{1 \times 2} + \cdots + \frac{1}{k(k + 1)} = \frac{k}{k + 1} \\
        \implies & \frac{1}{1\times 2} + \cdots + \frac{1}{k(k + 1)} + \frac{1}{(k + 1)(k + 2)} 
        = \frac{k}{k + 1} + \frac{1}{(k +1)(k + 2)} \\
        \implies & \frac{1}{1 \times 2} + \cdots + \frac{1}{k(k + 1)} + \frac{1}{(k + 1)(k + 2)}
        = \frac{k(k + 2) + 1}{(k + 1)(k + 2)} \\
        \implies & \frac{1}{1 \times 2} + \cdots + \frac{1}{k(k + 1)} + \frac{1}{(k + 1)(k + 2)}
        = \frac{k^{2} + 2k + 1}{(k + 1)(k + 2)} \\
        \implies & \frac{1}{1 \times 2} + \cdots + \frac{1}{k(k + 1)} + \frac{1}{(k + 1)(k + 2)}
        = \frac{k + 1}{k + 2}
    \end{align*}
    Therefore by the induction principle the statement must be true for all $n \in \mathbb{N}$.
\end{proof}

\begin{exercise}
    Show that for all $n \in \mathbb{N}$
    \[
        \frac{1}{1 \times 2 \times 3} + \frac{1}{2 \times 3 \times 4} + \cdots + \frac{1}{n(n + 1)(n + 2)} 
        = \frac{n^{2} + 3n}{4(n + 1)(n + 2)}
    \]
\end{exercise}
\begin{exercise}
    Show that for all $n \in \mathbb{N}$
    \[
        1\times 1! + 2\times 2! + \cdots + n\times n! = (n + 1)! - 1
    \]
\end{exercise}

\begin{example}[BDMO]
    Let $f\colon \mathbb{R} \to \mathbb{R}$ be a function such that $f(1) = 1$ and for any $x\in \mathbb{R}$, 
    $f(x+7)\geq f(x) + 7$ and $f(x + 1)\leq f(x) + 1$. Find the value of $f(2013)$.
\end{example}
\begin{sol}
    Notice that 
    \begin{align*}
        & f(x + 2) \leq f(x + 1) + 1 \leq f(x) + 2, \\
        & f(x + 3) \leq f(x + 2) + 1 \leq f(x) + 3
    \end{align*}
    We can generalize and say that if $n$ is a non-negative integer then $f(x + n) \leq f(x) + n$. 
    But how do we prove this? Let's try using induction!\\
    The question already tells us that the statement is true for $n=1$. We just need to show 
    that 
    \[
        f(x + k)\leq f(x) + k \implies f(x + k + 1)\leq f(x) + k + 1
    \]
    This is quite trivial.
    \[
        f(x + k + 1 )\leq f(x + k) + 1 \leq f(x) + k + 1
    \]
    And so we just proved that $f(x+n) \leq f(x) + n$. Now setting $n=7$, we get 
    \[
        f(x + 7) \leq f(x) + 7
    \]
    Therefore since $f(x+7) \geq f(x)+7$ and $f(x+7) \leq f(x) + 7$, we must have $f(x + 7) = f(x) + 7$. 
    That's great! But now what? Setting $x=1$, we get $f(8) = f(1) + 7 = 8$. But how can we find the value of $f(2013)$? 
    We can guess that $f(x) = x$ for all $x$. Can we prove this? If we can somehow show that $f(x) + 1 = f(x+1)$ for all $x$ 
    then we'll be able to easily show that $f(n) = n$ for all non-negative integer $n$. So let's try to prove that 
    $f(x) + 1 = f(x + 1)$ for all $x$. \\
    Suppose for the sake of contradiction that there exists some real number $r$ 
    such that $f(r) + 1 \neq f(r + 1)$. Therefore $f(r+1)$ must be 
    less than $f(r) + 1$. Now suppose $r = x + 6$ where $x$ is a real number. 
    \begin{align*}
        f(r + 1) < f(r) + 1 & \implies f(x+7) < f(x + 6) + 1 \\
                            & \implies f(x+7) < f(x) + 7
    \end{align*}
    But that is impossible as we've shown that $f(x + 7) = f(x) + 7$  for all $x \in \mathbb{R}$. 
    Hence such a real number $r$ cannot exist which implies $f(x) + 1 = f(x + 1)$ for all $x \in \mathbb{R}$. \\
    We can now use induction to show that $f(n) = n$ for all non-negative integer $n$. For $n=1$ the statement 
    is true. We need to prove that if $f(k) = k$ then $f(k+1) = k + 1$.
    \[
        f(k + 1) = f(k) + 1 \implies f(k + 1) = k + 1
    \]
    And we are done! YAY! We not only found the value of $f(2013)$ but also found the value of $f(n)$ for all 
    non-negative integer $n$. Awesome, right?
\end{sol}

\begin{example}
    If $n$ is a non-negative integer and $x, y$ are two real numbers then 
    \[
        (x + y)^{n} = \sum_{k=0}^{n} \binom{n}{k} x^{k}y^{n-k}
    \]
\end{example}
\begin{proof}
    The statement is evidently true for $n=1$. Now we need to show that 
    \[
        (x + y)^{m} = \sum_{k=0}^{m} \binom{m}{k} x^{k} y^{m - k}
        \implies (x + y)^{m + 1} = \sum_{k=0}^{m + 1} \binom{m+1}{k} x^{k} y^{m - k + 1}
    \]
    Okay, let's try to prove it!
    \begin{align*}
        (x + y)^{m} \times (x + y) 
        &= (x + y)^{m} x + (x + y)^{m} y \\
        &= \sum_{k=0}^{m} \binom{m}{k} x^{k + 1} y^{m - k} + \sum_{k=0}^{m} \binom{m}{k} x^{k} y^{m - k + 1}
    \end{align*}
    Now 
    \begin{align*}
        \sum_{k=0}^{m} \binom{m}{k} x^{k + 1} y^{m - k} &= x^{m + 1} + 
        \sum_{k=0}^{m-1} \binom{m}{k} x^{k+1} y^{m - k} \\
        &= x^{m+1} + \sum_{k=1}^{m} \binom{m}{k-1} x^{k} y^{m - k + 1} \\ 
        \sum_{k=0}^{m} \binom{m}{k} x^{k} y^{m - k + 1} &= y^{m + 1} + 
        \sum_{k = 1}^{m} \binom{m}{k} x^{k} y^{m - k + 1}
    \end{align*}
    Therefore 
    \begin{align*}
        (x + y)^{m + 1} &= \sum_{k=0}^{m} \binom{m}{k} x^{k + 1} y^{m - k} + \sum_{k=0}^{m} \binom{m}{k} x^{k} y^{m - k + 1} \\
        &= x^{m + 1} + \sum_{k=1}^{m} \left\{\binom{m}{k} + \binom{m}{k-1} \right\} x^{k} y^{m - k + 1} + y^{m + 1}
    \end{align*}
    Now 
    \begin{align*}
        \binom{m}{k} + \binom{m}{k - 1} &= \frac{m!}{(m - k)!k!} + \frac{m!}{(m - k + 1)!(k - 1)!} \\
                                        &= \frac{m!}{(m - k)!(k - 1)!} \left(\frac{1}{k} + \frac{1}{m - k + 1}\right) \\
                                        &= \frac{m!}{(m - k)!(k - 1)!}\left(\frac{m + 1}{k(m - k + 1)}\right) \\
                                        &= \frac{(m + 1)!}{(m - k + 1)! k!} \\
                                        &= \binom{m + 1}{k}
    \end{align*}
    Thus we have 
    \begin{align*}
        (x + y)^{m + 1} 
        &= x^{m + 1} + \sum_{k=1}^{m} \left\{\binom{m}{k} + \binom{m}{k-1} \right\} x^{k} y^{m - k + 1} + y^{m + 1} \\
        &= x^{m + 1} + \sum_{k=1}^{m} \binom{m + 1}{k} x^{k} y^{m - k + 1} + y^{m + 1} \\
        &= \sum_{k = 0}^{m + 1} \binom{m + 1}{k} x^{k} y^{m + 1 - k}
    \end{align*}
    Voila! We just proved the binomial theorem using induction!
\end{proof}

\begin{example}
    Show that if $H_{n}$ is the $n$-th Harmonic Number, where $n \in \mathbb{N}$, then 
    \[
        1 + \frac{\floor{\log_{2} n}}{2} \leq H_{n}
    \]
\end{example}
\begin{proof}
    The problem looks quite complicated! So let's try proving a more simplified version of the problem first. 
    We will show that if $n$ is a positive integer then
    \[
        1 + \frac{n}{2} \leq H_{2^{n}}
    \]
    This looks simpler than the original problem because it does not contain the ugly floor function. 
    But how do we prove this? We are going to use induction on $n$. \\
    The base case, $n = 1$, is true. We just have to show that 
    \[
        1 + \frac{k}{2} \leq H_{2^{k}} \implies 1 + \frac{k + 1}{2} \leq H_{2^{k + 1}}
    \]
    Notice that 
    \[
        H_{2^{k + 1}} = H_{2^{k}} + \frac{1}{2^{k} + 1} + \frac{1}{2^{k} + 2} + \cdots + \frac{1}{2^{k} + 2^{k}}
    \]
    Now since 
    \[
        2^{k + 1} \geq 2^{k + 1} - 1 \geq \cdots \geq 2^{k} + 2 \geq 2^{k} + 1
    \]
    we have 
    \begin{align*}
        \frac{1}{2^{k + 1}} \leq \frac{1}{2^{k + 1} - 1} \leq \cdots \leq \frac{1}{2^{k} + 2} \leq \frac{1}{2^{k} + 1}
    \end{align*}
    Adding the inequalities we get 
    \begin{align*}
        \frac{2^{k}}{2^{k + 1}} &\leq \frac{1}{2^{k + 1}} + \cdots + \frac{1}{2^{k} + 2} + \frac{1}{2^{k} + 1} \\
        \implies \frac{1}{2} &\leq \frac{1}{2^{k + 1}} + \cdots + \frac{1}{2^{k} + 2} + \frac{1}{2^{k} + 1}
    \end{align*}
    Therefore 
    \[
        H_{2^{k}} + \frac{1}{2} \leq H_{2^{k+1}} \implies 1 + \frac{k + 1}{2} \leq H_{2^{k + 1}}
    \]
    And we are done! No, not really. We still have to solve the original problem. \\
    Suppose that $\floor{\log_{2} n} = k$. Now
    \begin{align*}
        2^{k} \leq n & \implies H_{2^{k}} \leq H_{n} \\
                     & \implies 1 + \frac{k}{2} \leq H_{n} \\
                     & \implies 1 + \frac{\floor{\log_{2} n}}{2} \leq H_{n}
    \end{align*}
\end{proof}

\section{Variants of the Induction Principle}

\begin{theorem}\label{induction-mul-bases}
    If $\mathcal{P}(n)$ is some predicate and if 
    \begin{itemize}
        \ii $\mathcal{P}(n_{0}), \mathcal{P}(n_{0} + 1), \cdots, \mathcal{P}(n_{0} + m)$ are all true and 
        \ii $\mathcal{P}(k) \wedge \mathcal{P}(k + 1) \wedge \cdots \wedge \mathcal{P}(k + m)$ implies $\mathcal{P}(k + m + 1)$
    \end{itemize}
    then $\mathcal{P}(n)$ is true for all integers $n\geq n_{0}$.
\end{theorem}
\begin{proof}
    Let $Q(n)$ be the predicate
    \[
        Q(n) : \mathcal{P}(n) \wedge \mathcal{P}(n + 1) \wedge \cdots \wedge \mathcal{P}(n + m)
    \]
    The base case $Q(n_{0})$ is true. Assume that $Q(k)$ is true. Now 
    \begin{align*}
                 &\mathcal{P}(k) \wedge \mathcal{P}(k + 1) \wedge \cdots \wedge \mathcal{P}(k + m) 
                  \rightarrow \mathcal{P}(k + m + 1) \\
        \implies &\mathcal{P}(k) \wedge \mathcal{P}(k + 1) \wedge \cdots \wedge \mathcal{P}(k + m) 
        \rightarrow \mathcal{P}(k + 1) \wedge \cdots \wedge \mathcal{P}(k + m) \wedge \mathcal{P}(k + 1 + m) \\
        \implies & Q(k) \rightarrow Q(k + 1)
    \end{align*}
    Therefore by the induction principle $Q(n)$ must be true for all $n\geq n_{0}$. If $Q(n)$ is true for all 
    $n \geq n_{0}$ then $\mathcal{P}(n)$ must also be true for all $n\geq n_{0}$.
\end{proof}

\begin{example}
    Find the general term of the sequence defined by $x_{0}=3, x_{1}=4$ and 
    \[
        x_{n+1} = x_{n-1}^{2} - n x_{n}
    \]
    for all $n\geq 1$.
\end{example}
\begin{proof}
    The first few values of the sequence are 
    \[
        3, 4, 5, 6, 7, \cdots
    \]
    We can guess that $x_{n} = n + 3$. We will use induction on $n$ to show that 
    $x_{n} = n + 3$ for all $n \geq 0$. \\
    Notice that normal induction won't work here because we need the values of both $x_{k}$ and $x_{k-1}$ 
    to find the value of $x_{k + 1}$. But don't worry, we can use Theorem \ref{induction-mul-bases}. \\
    Let $P(n)$ be the predicate 
    \[
        P(n): x_{n} = n + 3
    \]
    Since $x_{0} = 3, x_{1} = 4$, $P(0)$ and $P(1)$ are both true. We need to show that if $k\geq 0$ 
    is an integer then 
    \[
        P(k) \wedge P(k + 1) \implies P(k + 2)
    \]
    Assume that $P(k)$ and $P(k + 1)$ are both true. Now
    \begin{align*}
                 & x_{k + 2} = x_{k}^{2} - (k + 1)x_{k + 1} \\
        \implies & x_{k + 2} = (k + 3)^{2} - (k + 1)(k + 4)^{2} \\
        \implies & x_{k + 2} = k^{2} + 6k + 9 - k^{2} - 5k - 4 \\
        \implies & x_{k + 2} = k + 5 = (k + 2) + 3
    \end{align*}
    Therefore by Theorem \ref{induction-mul-bases}, we can say that $P(n)$ is true for all $n \geq 0$. 
    Hence $x_{n} = n + 3$ for all $n \geq 0$.
\end{proof}

\begin{example}[Binet's Formula]
    Let $F_{n}$ be the $n$-th Fibonacci number. Show that 
    \[
        F_{n} = \frac{\phi^{n} - \psi^{n}}{\phi - \psi}
    \]
    where $\phi = \frac{1 + \sqrt{5}}{2}$ and $\psi = \frac{1 - \sqrt{5}}{2}$ are the two real roots 
    of the quadratic equation $x^{2} - x - 1$.
\end{example}
\begin{proof}
    We will use induction on $n$. The base cases $n = 0$ and $n = 1$ are both true.
    \begin{align*}
        & F_{0} = \frac{\phi^{0} - \psi^{0}}{\phi - \psi} \implies 0 = 0 \\
        & F_{1} = \frac{\phi^{1} - \psi^{1}}{\phi - \psi} \implies 1 = 1
    \end{align*}
    Assume the formula works for $k$ and $k + 1$ where $k \geq 0$ is some integer. Now 
    \begin{align*}
        F_{k + 2} & = F_{k + 1} + F_{k} \\
                  & = \frac{\phi^{k + 1} - \psi^{k + 1}}{\phi - \psi} + \frac{\phi^{k} - \psi^{k}}{\phi - \psi} \\
                  & = \frac{\left(\phi^{k + 1} + \phi^{k}\right) - \left(\psi^{k + 1} + \psi^{k}\right)}{\phi - \psi} \\
                  & = \frac{\phi^{k}\left(\phi + 1\right) - \psi^{k}(\psi + 1)}{\phi - \psi} \\
                  & = \frac{\phi^{k + 2} - \psi^{k + 2}}{\phi - \psi}
    \end{align*}
    Therefore the formula works for all integers $n \geq 0$.
\end{proof}
\end{document}
