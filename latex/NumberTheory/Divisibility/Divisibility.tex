\chapter{Divisibility}
\section{Divisibility Properties}

\begin{theorem}[Division Algorithm]
    For any integers $a,b$, with $b>0$, there exists \textbf{unique} integers 
    $q$ and $r$ such that,
    \[
        a = qb + r, \quad 0 \leq r < b
    \]
\end{theorem}
\begin{proof}
    Suppose $\mathcal{S} = \left\{ a-qb \mid q \in \mathbb{Z}, a-qb \geq 0 \right\}$. 
    We want to show that $r$ is the least element of the set $\mathcal{S}$. 
    But first we have to show that $\mathcal{S}$ is a non-empty set. Notice, 
    \[
    \floor{\frac{a}{b}} \leq \frac{a}{b} 
    \implies \floor{\frac{a}{b}} \times b \leq a 
    \implies 0 \leq a - \floor{\frac{a}{b}}b
    \]
    Therefore for the choice $q = \floor{\frac{a}{b}}$, $a - \floor{\frac{a}{b}}b \in \mathcal{S}$. 
    Thus $\mathcal{S}$ is a non-empty set of non-negative integers.
    \begin{theorem*}[Well Ordering Principle]
        Every non-empty set of non-negative integers contains a least element. 
        That is if $\mathcal{S}$ is a non-empty set of non-negative integers then 
        there exists a non-negative integer $n \in \mathcal{S}$ such that 
        $n \leq x$ for every $x \in \mathcal{S}$.
    \end{theorem*}
    Now from the Well Ordering Principle we know that $\mathcal{S}$ contains a least element. 
    Let $r$ be the least element of $\mathcal{S}$. Assume $r \geq b$ and let $r' = r - b$. 
    Since $r \geq b$ we have,
    \[
        r' = r - b \geq 0 
        \implies r' = a - qb - b = a - (q+1)b \geq 0 
        \implies r' \in \mathcal{S}
    \]
    But this contradicts our assumption that $r$ is the least element of $\mathcal{S}$. 
    Thus $r$ must be less than $b$. Now we will prove the uniqueness of the integers $r$ and $q$. 
    Suppose there exists integers $q'$ and $r'$, with $0\leq r' < b$, such that $a = q'b + r'$. Now, 
    \[
        q'b + r' = qb + r 
        \implies (q' - q)b = r - r' 
        \implies \abs{q' - q} b = \abs{r - r'}
    \]
    Adding the two inequalities, $0 \leq r < b$ and $-b < -r' \leq 0$, 
    we get $-b < r - r' < b \implies 0 \leq \abs{r-r'} < b$. Therefore,
    \[
    0 \leq \abs{r-r'} < b 
    \implies 0\leq \abs{q' - q} b < b 
    \implies 0\leq \abs{q' - q} < 1
    \]
    Since $\abs{q' - q}$ is a non-negative integer we must have $q' - q=0 \implies q= q'$ 
    and which in turn implies $r=r'$.
\end{proof}
\begin{corollary}
    If $a$ and $b$ are integers, with $b\neq 0$, then there exists unique integers $q$ and $r$ such that,
    \[
        a = qb + r, \quad 0\leq r < \abs{b}
    \]
\end{corollary}
