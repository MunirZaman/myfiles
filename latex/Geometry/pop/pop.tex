\documentclass[11pt,numbers=noenddot,svgnames,dvipsnames]{scrartcl}
\usepackage[top=1in, left=1in, right=1in, bottom=1in]{geometry}
\usepackage[asy]{munir}
\DeclareMathOperator{\pow}{Pow}

\title{Power of a Point}
\author{Munir Uz Zaman}
\date{Date: \today}

\begin{document}
\maketitle

\section{Power of a Point}

\begin{theorem}[Power of a Point]
    Consider a circle $\omega$ and an arbitrary point $P$. 
    The \vocab{power of $P$} with respect to circle $\omega$ 
    is defined by 
    \[
        \pow_{\omega}(P) = OP^{2} - r^{2}
    \]
    where $O$ is the center of $\omega$ and $r$ is the radius.
    \begin{itemize}
        \ii $\pow_{\omega}(P)$ is negative, zero or positive according to whether 
        $P$ is inside, on or outside the circle $\omega$, respectively. 
        \ii If a line $\ell$ through point $P$ intersects $\omega$ at points $X$ 
        and $Y$ then 
        \[
            PX \times PY = \abs{\pow_{\omega}(P)}
        \]
        \ii If the line $PA$ is tangent to circle $\omega$ at point $A$ then 
        \[
            PA^{2} = \pow_{\omega}(P)
        \]
    \end{itemize}
\end{theorem}

\begin{theorem}[Converse of the Power of a Point]
    Let $A, B, X, Y$ be four distinct points in the plane and let 
    $P = \ol{AB} \cap \ol{XY}$ be the intersection point of lines 
    $\ol{AB}$ and $\ol{XY}$. If $PA \times PB = PX \times PY$ and 
    if $P$ lies on both segments $AB$ and $XY$, or in neither segment 
    then the points $A, B, X, Y$ are concyclic.
\end{theorem}

\end{document}
